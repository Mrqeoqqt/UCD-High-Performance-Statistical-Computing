\documentclass[a4paper,11pt,onecolumn,twoside]{article}
\usepackage{fancyhdr}
\usepackage{amsmath,amsfonts,amssymb}
\usepackage{graphicx}
\usepackage{fontspec}
\usepackage{booktabs}
\usepackage{indentfirst}
\usepackage{enumitem}
\usepackage{subfigure}
\usepackage{float}
\usepackage{caption}
\usepackage{graphicx}
% Please change the following fonts if they are not available.
\addtolength{\topmargin}{-54pt}
\setlength{\oddsidemargin}{-0.9cm}
\setlength{\evensidemargin}{\oddsidemargin}
\setlength{\textwidth}{17.00cm}
\setlength{\textheight}{24.50cm}
\usepackage{listings}
\usepackage{xcolor}
\usepackage[colorlinks,linkcolor=black,anchorcolor=black,citecolor=black]{hyperref}
\usepackage{ctex}
\definecolor{dkgreen}{rgb}{0,0.6,0}
\definecolor{mauve}{rgb}{0.58,0,0.82}

\lstset{
	numbers=left,
	numberstyle=\tiny,
	stringstyle=\color{purple},
	basicstyle=\footnotesize\ttfamily, 
	keywordstyle=\color{blue}\bfseries,
	commentstyle=\color{olive},
	frame=shadowbox,
	%framerule=0pt,
	%backgroundcolor=\color{pink},
	rulesepcolor=\color{red!20!green!20!blue!20},
	%rulesepcolor=\color{brown}
	%xleftmargin=2em,xrightmargin=2em,aboveskip=1em
	escapeinside=``, 
	basicstyle=\tiny
}
\renewcommand{\baselinestretch}{1.1}
\parindent 22pt

\title{\Large \textbf{Final Project Proposal}}
\author{
%Wangqian Miao\footnote{Two authors are both exchange students from Nanjing University.}
%\\[2pt]
%{\large \textit{Kuang Yaming Hornors School, Biophysics, Nanjing University}}\\[6pt]
%Mingyi Xue\\[2pt]
%{\large \textit{School of Chemistry and Chemical Engineering, Nanjing University}}\\[6pt]
%Rui Wang\\[2pt]
%{\large \textit{Business School, Nanjing University}}\\[6pt]
\textbf{Instructor}: Prof. Cho-Jui Hsieh\\[2pt]
{\large \textit{Department of Statistics, University of California, Davis}}\\[2pt]
}
\date{\today}

\fancypagestyle{firststyle}
{
   \fancyhf{}
   \fancyhead[C]{STA141C: Big Data $\&$ High Performance Statistical
   	Computing, Course Project}
   \fancyhead[R]{\thepage}
}

\pagestyle{fancy}
\fancyhf{}
\fancyhead[LE,RO]{\thepage}
\fancyhead[CE]{STA141C: Big Data $\&$ High Performance Statistical
	Computing}
\fancyhead[RE]{Final Project}
\fancyhead[CO]{W. Miao, M. Xue, R.Wang: Kobe Bryant Shot Selection Using Classification}
\fancyhead[LO]{}
\setlist{nolistsep}
\captionsetup{font=small}
\newcommand{\supercite}[1]{\textsuperscript{\cite{#1}}}
\begin{document}
\maketitle
\thispagestyle{firststyle}
\setlength{\oddsidemargin}{ 1cm}
\setlength{\evensidemargin}{\oddsidemargin}
\setlength{\textwidth}{15.50cm}
\vspace{-.8cm}
\setcounter{page}{1}
\setlength{\oddsidemargin}{-.5cm}  % 3.17cm - 1 inch
\setlength{\evensidemargin}{\oddsidemargin}
\setlength{\textwidth}{17.00cm}
%\tableofcontents
%\newpage
\section{Members}
\begin{table}[H]
	\centering
	\begin{tabular}{cc}
		\hline
		\textbf{Names} & \textbf{Email Address}\\
		\hline
		Wangqian Miao&\texttt{wqmiao@ucdavis.edu}\\
		Mingyi Xue & \texttt{myxue@ucdavis.edu}\\
		Rui Wang&\texttt{ruiang@ucdavis.edu} \\
		\hline
	\end{tabular}
\end{table}
\section{Project}
\begin{table}[H]
	\centering
	\caption*{\textbf{Project Infomation}}
	\begin{tabular}{cc}
		\midrule[1.5pt]
		\textbf{Topic}& Kobe Bryant Shot Selection\\	
		\textbf{Input Format}&tabular data\\
		\textbf{Output}&binary classification\\
		\textbf{Tools}&pandas, sklearn\\
		\textbf{Algorithms}&logistic regression, SVM, neural networks\\
		\midrule[1.5pt]
	\end{tabular}
\end{table}
\begin{enumerate}
	\item The \textbf{dataset} is from \url{https://www.kaggle.com/c/kobe-bryant-shot-selection/data} .
	\item Our \textbf{goal} is to perform varied classification algotithms mentioned above to predict which shots Kobe sank, comparing the efficiency and accuracy of these methods.\par
	\item One \textbf{difficulty} we will confront is \textit{feature engnieering}. Because this dataset involves $25$ explanatory variables, types of which contain numeric, categoric and datetime, we are supposed to deal with different types of feature. Firstly, after we transform categorical variables to dummy variables, what to do if there are too many variables after this transformation. Secondly, how to deal with datetime variables, whether to treat them as categorical variables or not.\par
	\item Another \textbf{difficulty }is memory capacity. Since the training dataset has more than 30 thousand samples with 25 features each, will personal computer consumes excessive time running classification algorithms on the dataset?\par
\end{enumerate}
\section{Reference}
\begin{enumerate}
	\item \url{https://dnc1994.com/2016/04/rank-10-percent-in-first-kaggle-competition/}
	\item \url{https://www.zhihu.com/question/23987009}
	\item \url{http://www.cnblogs.com/jasonfreak/p/5448385.html}
	\item 机器学习\,\,周志华
\end{enumerate}




\end{document}